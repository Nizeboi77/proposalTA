%\let\clearpage\relax% Don't allow page break
\begin{center}
\noindent \huge Sistem Monitoring dan Otomasi Proses Sipon berbasis IoT dengan menggunakan ESP32
\end{center}

\section{PENDAHULUAN}

% Ubah bagian-bagian berikut dengan isi dari pendahuluan

\subsection{Latar Belakang}
\label{sec:latarbelakang}

Udang merupakan salah satu komoditas unggulan dunia, yang berarti komoditas tersebut mempunyai prospek yang besar pada sector perikanan, hal tersebut tentunya dapat meningkatkan devisa negara melalui ekspor komoditas perikanan. Tingginya permintaan udang dari dalam dan luar negeri memjadikan Indonesia sebagai salah satu pengirim udang terbesar di dunia[2]. Adapun salah satu jenis udang yang paling banyak diminati di dunia adalah Udang Vanname. Alasannya antara lain adalah Udang Vanname mampu hidup di perairan yang memiliki salinitas rendah sampai tinggi, dan juga dapat beradaptasi dengan lingkungan yang bersuhu rendah, serta mempunyai kelangsungan hidup yang tinggi[3]. Jenis Udang Vaname memiliki nafsu makan yang relative tinggi dan mempunyai kemampuan untuk memanfaatkan pakan dengan kadar protein yang rendah sehingga pemberian pola pakan dapat disesuaikan dengan budidaya tambak[5]. Kemampuan tersebut yang menimbulkan kecocokan bagi pengusaha tambak untuk membudidayakan jenis udang tersebut[4]. \\

Suhu dan pH merupakan beberapa indicator dari keberlanjutan budidaya udang vaname sehingga penting untuk dipantau. Untuk indikator suhu, penting dilakukan pengukuran untuk mengetahui karakteristik perairan, yang mana indikator tersebut adalah faktor abiotik yang memegang peranan penting bagi kehidupan organisme di perairan, Adapun suhu optimal bagi udang berkisar antara 29-32°C[12]. Pada indicator pH, pengupayaan untuk mempertahankan PH air dalam budidaya tambak udang menjadi suatu kewajiban supaya kualitas air dapat terjaga dengan baik dan stabil. Selain itu, konsentrasi pH air berpengaruh terhadap nafsu makan udang serta reaksi kimia dalam air, dan juga apabila konsentrasi pH air dibawah batas toleransi, udang akan menjadi sulit untuk mengganti kulit yang mana akan menjadi lembek sehingga sintasan rendah[13]. pH optimal bagi udang berkisar antara 7,0 – 8,5[12]. \\

Pada sistem budidaya udang vaname, limbah sisa yang mengendap di dasar kolam mempengaruhi kualitas air. Oleh karena itu , terdapat kebutuhan untuk membersihkan residu ini secara teratur untuk menjaga kualitas air dalam kondisi baik. Proses pembersihan residu dinamakan Proses Sipon. Proses sipon pada umumnya mengambil kotoran (sisa pakan maupun feses) dari dasar tambak atau kolam udang dengan menggunakan selang yang menyedot kotoran. \\

Seiring berkembangnya zaman, tidak sedikit pekerjaan manusia yang terbantu dengan adanya teknologi, baik berupa mesin ataupun robot yang mampu bekerja secara otomatis. Sama halnya pada bidang perikanan, khususnya budidaya, contoh yang umum adalah seperti mesin pakan otomatis, yang mana mesin tersebut mampu memberikan pakan secara otomatis dengan mengatur jadwal yang telah ditentukan oleh pengusaha tambak[6]. Pekerja diberikan fasilitas control tersebut dengan tujuan untuk memudahkan dalam pemberian pakan udang, yang mana pakan udang yang diberikan disesuaikan dengan kondisi kebutuhan pakan udang (shrimp feed requirements) agar endapan dari sisa pakan di dasar tidak terlalu banyak[7]. \\

Dalam tugas akhir ini juga melakukan hal yang sama yakni mempermudah pekerjaan manusia khususnya petambak udang dalam membersihkan dasar tambak udang dari endapan sisa makanan yang dapat menyebabkan penurunan kualitas air, dan juga untuk mengawasi kualitas air tambak yang sama dengan melakukan monitoring melalui website. Sistem dan Alat yang dibuat akan menggantikan pekerjaan pembersihan dasar tambak yang pada umumnya dilakukan oleh pekerja tambak yang menyedot kotoran di tambak secara langsung seperti yang dijelaskan sebelumnya, serta mempermudah dalam pemantauan kualitas air tambak. \\



\subsection{Permasalahan}
\label{sec:permasalahan}

Berdasarkan latar belakang tersebut dapat diketahui bahwa untuk mengetahui status keberlanjutan budidaya tambak udang vaname, diperlukan pemantauan suhu dan pH air, selain itu sebagian besar proses sipon masih menggunakan cara yang tradisional yang mana membutuhkan waktu yang lama dan hasil yang tentunya tidak maksimal. \\

\noindent Oleh karena itu, diperlukan sebuah sistem untuk melakukan manajemen otomasi dan monitoring kualitas air sehingga proses sipon dapat dilakukan secara efektif dan efisien.

\subsection{Penelitian Terkait}
\label{sec:Penelitian Terkait}

Dalam penelitian yang dilakukan oleh saudara Indra Jaya dan M Iqbal dari Marine Science and Technology, Faculty of Fisheries and Marine Sciences, IPB University[14]. Dilakukan pembuatan alat yang melakukan proses sipon secara otomatis, yang mana instrument yang digunakan memanfaatkan prinsip equilibrium antara alat utama yang berbentuk piramida dengan alat bantu berupa penyimpan air (water container).

\subsection{Gap Penelitian}
\label{sec:Gap Penelitian}

Dalam penelitian yang dilakukan oleh saudara Indra Jaya dan M Iqbal[14], belum dilakukan pengintegrasian antara alat dengan end device, baik  berupa web yang mana dapat melakukan pengaturan mengenai berapa lama waktu sipon, jadwal sipon yang dilakukan, serta monitoring kualitas air.

\subsection{Tujuan Penelitian}
\label{sec:Tujuan Penelitian}

Membuat sistem yang melakukan monitoring kualitas air dan proses sipon secara otomatis dan dapat dipantau serta diatur melalui web.
