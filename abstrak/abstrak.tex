\begin{center}
  \large\textbf{ABSTRAK}
\end{center}

\vspace{2ex}

\begingroup
  % Menghilangkan padding
  \setlength{\tabcolsep}{0pt}

  \noindent
  \begin{tabularx}{\textwidth}{l >{\centering}m{2em} X}
    % Ubah kalimat berikut dengan nama mahasiswa
    Nama Mahasiswa    &:& Achmad Pahlevy Aminullah Nizaruddin \\
    NRP	&:& 0721 18 4000 0001 \\
    Semester	&:& Ganjil 2021 / 2022 \\

    % Ubah kalimat berikut dengan judul tugas akhir
    Judul Tugas Akhir &:&	Sistem Monitoring dan Otomasi Proses Sipon Berbasis IoT dengan menggunakan ESP32 Untuk  Tambak Udang Vaname \\
	& & \textbf{\emph{Monitoring and Automated Syphon System Based On IoT with ESP32 For Vaname Shrimp Pond}} \\

    % Ubah kalimat-kalimat berikut dengan nama-nama dosen pembimbing
    Pembimbing        &:& 1. Arief Kurniawan, S.T., M.T. \\
                      & & 2. Dion Hayu Fandiantoro , S.T., M.Eng. \\
    \underline{Uraian Tugas Akhir}	&:&
  \end{tabularx}
\endgroup


% Ubah paragraf berikut dengan abstrak dari tugas akhir
\vspace{0.5cm}
\noindent
Pada sistem budidaya udang, suhu air dan pH merupakan beberapa indikator penting dalam keberlanjutan budidaya tersebut sehingga penting untuk dipantau. Selain itu, limbah sisa yang mengendap di dasar kolam mempengaruhi kualitas air. Oleh karena itu , terdapat kebutuhan untuk membersihkan residu ini secara teratur untuk menjaga kualitas air dalam kondisi baik. Sampai saat ini, kebanyakan proses sipon masih menggunakan cara tradisional sehingga membutuhkan waktu yang relatif lama. Maka dari itu, diperlukan sistem yang dapat melakukan proses sipon secara otomatis dan juga melakukan monitoring kualitas air, kedua kemampuan pada sistem tersebut bertujuan untuk mempermudah proses sipon serta pemantauan kualitas air. Sistem ini menggunakan microcontroller ESP32 yang menggunakan komunikasi WiFi, kemudian hasil komunikasi atau output yang berupa data sensor  akan disalurkan ke website. Adapun website dalam sistem ini berfungsi untuk menampilkan hasil monitoring kualitas air serta mengatur otomasi proses sipon seperti pengaturan jadwal sipon, durasi sipon, dan sejenisnya.

\vspace{1.5cm}
\begin{center}
  \begin{minipage}{.45\linewidth}
    \begin{flushleft}
	\begin{center}
      \textbf{Dosen Pembimbing I} \\
      \vspace{1.5cm}
       \underline{ Arief Kurniawan, S.T., M.T.} \\
	NIP. 1197409072002121001
	\end{center} 
    \end{flushleft}
  \end{minipage}
  \hfill
  \begin{minipage}{.45\linewidth}
    \begin{flushleft}
	\begin{center}
      \textbf{Dosen Pembimbing II} \\
      \vspace{1.5cm}
       \underline{Dion Hayu Fandiantoro , S.T., M.Eng.} \\
	NPP. 1994202011064
	\end{center} 
    \end{flushleft}
  \end{minipage}
\end{center}

\vspace{1cm}

\begin{center}
  \begin{minipage}{.50\linewidth}
	\begin{center}
      \textbf{Mengetahui,} \\
	\textbf{Departemen Teknik Komputer FTEIC - ITS} \\
	\textbf{Kepala,}\\
      \vspace{1.5cm}
       \underline{ Dr. Supeno Mardi Susiki Nugroho, S.T., M.T.} \\
	NIP. 197003131995121001
	\end{center} 
  \end{minipage}
\end{center}